%%% Ukázka použití některých konstrukcí LaTeXu

\subsection{Ukázka \LaTeX{}u}
\label{ssec:ukazka}

This short subsection serves as an~example of basic \LaTeX{} constructs,
which can be useful for writing a~thesis.

Let us start with lists:

\begin{itemize}
\item The logo of Matfyz is displayed in figure~\ref{fig:mff}.
\item This is subsection~\ref{ssec:ukazka}.
\item Citing literature~\cite{lamport94}.
\end{itemize}

Different kinds of dashes:
red-black (short),
pages 16--22 (middle),
$45-44$ (minus),
and this is --- as you could have expected --- a~sentence-level dash,
which is the longest.
(Note that we have follwed \verb|a| by a~tilde instead of a~space
to avoid line breaks at that place.)

\newtheorem{theorem}{Theorem}
\newtheorem*{define}{Definition}	% Definice nečíslujeme, proto "*"

\begin{define}
A~{\sl Tree} is a connected graph with no cycles.
\end{define}

\begin{theorem}
This theorem is false.
\end{theorem}

\begin{proof}
False theorems do not have proofs.
\end{proof}

\begin{figure}
	\centering
	\includegraphics[width=30mm]{../img/logo.eps}
	\caption{Logo of MFF UK}
	\label{fig:mff}
\end{figure}
